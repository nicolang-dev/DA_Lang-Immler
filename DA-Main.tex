\documentclass[]{article}

\usepackage[utf8]{inputenc}
\usepackage[T1]{fontenc}
\usepackage{lmodern}
\usepackage[ngerman]{babel}
\usepackage{amsmath}
\usepackage{tabularx}

\usepackage{graphicx}
\graphicspath{ {./images/} }

\title{\textbf{Multi-Room Sound Adapter}}
\author{Nico Lang, Philipp Immler}
\date{Februar 2025}

\begin{document}

\maketitle
\thispagestyle{empty}

\pagebreak

\section{Projekt}
\subsection{Projektteam}
\textbf{Nico Lang}\\
Wirtschaftsingenieure/Betriebsinformatik\\
Grießau\\
6651 Häselgehr AT\\
Nico.Lang@hak-reutte.ac.at\\
\\
\textbf{Philipp Immler}\\
Wirtschaftsingenieure/Betriebsinformatik\\
Hoheneggweg 21a\\
6682 Vils AT\\
Philipp.Immler@hak-reutte.ac.at

\pagebreak

\subsection{Eidesstattliche Erklärung}
Hiermit versichere ich, dass ich die vorliegende Arbeit selbstständig verfasst und keine anderen Hilfsmittel als die angegebenen benützt habe. Die Stellen, die anderen Werken (gilt ebenso für Werke aus elektronischen Datenbanken oder aus dem Internet) wörtlich oder sinngemäß entnommen sind, habe ich unter Angabe der Quelle und Einhaltung der Regeln wissenschaftlichen Zitierens kenntlich gemacht. Diese Versicherung umfasst auch in der Arbeit verwendete bildliche Darstellungen, Tabellen, Skizzen und Zeichnungen. Für die Erstellung der Arbeit habe ich auch folgende Hilfsmittel generativer KI-Tools (ChatGPT 3.5) zu folgendem Zweck verwendet: Inspiration und allgemeine Information. Auch Übersetzer (DeepL) wurden zur Hilfe genommen. Die verwendeten Hilfsmittel wurden vollständig und wahrheitsgetreu inkl. Produktversion und Prompt ausgewiesen.\\

\vspace{30mm}

\noindent
\begin{minipage}[c]{5cm}
	\centering \dotfill \\
	Ort, Datum
\end{minipage}
\hfill
    \begin{minipage}[c]{5cm}
        \centering \dotfill \\
        Unterschrift Schüler/in
    \end{minipage}
    
\vspace{10mm}

\noindent
\begin{flushright}
    \begin{minipage}[c]{5cm}
        \centering \dotfill \\
        Unterschrift Schüler/in
    \end{minipage}
\end{flushright}

\pagebreak

\subsection{Abstract Deutsch}
\subsection{Abstract English}
\subsection{Danksagung}

\tableofcontents

\section{Planung}
\subsection{Festlegung Funktionsweise}
\subsection{Auswahl Hardwarekomponenten}
\subsection{Auswahl Technologien}
\subsection{Auswahl Softwaretools}

\section{Entwicklung}
\subsection{Design Platine}
\subsection{Bestückung Platine}
\subsection{Entwicklung Software Adapter}
\subsection{Entwicklung Smartphone-App}
\subsection{Design Adaptergehäuse}
\subsection{Fertigung Adaptergehäuse}

\section{Testen und Fehlerbehebung}
\subsection{Testen des Gesamtsystems}
\subsection{auftretende Fehler beheben}
\subsection{Test auf Cybersecurity}
\subsection{Auftretende Sicherheitslücken schließen}

\section{Einzelnachweise}
\subsection{Literaturverzeichnis}
\subsection{Abbildungsverzeichnis}
\subsection{Anhang}

\end{document}